\chapter{Background}
Necessary background for topic, in order to understand the problem, motivation and contribution. Approximately 10 pages.
\section{JSON}
\section{Javascript Framework D3}\label{sec:framework}
Eine grundlegende Entscheidungen zu Beginn der Arbeit betraf die Frage, ob ein Javascript-Framework zur Unterstützung bei der Baumvisualisierung eingesetzt werden sollte und, wenn ja, welches. Die Wahl fiel dabei auf D3, weil es neben Unterstützung von Baum- und Graphenstrukturen auch eingebaute Funktionen für Animationen aufweist, welche, wie im Kapitel \ref{sec:animation} beschrieben, eine zentrale Rolle für eine intuitive Bedienung spielen. D3 ist außerdem weit verbreitet, wodurch bei der Entwicklung auf eine reichhaltige Menge von Beispielen und Anregungen zurückgegriffen werden kann.
\subsection{Hierarchies}
\todo{Erklärung von heirarchies und wie man damit bäume speichern kann}
\subsection{Selections}
\todo{Erklärung von selections und speziell der Funktionen enter(), exit() und data()}
\section{SVG}
\todo{erläuterung von svg elementen, welche elemente im inneren benutzt werden können und ihre bedeutung}