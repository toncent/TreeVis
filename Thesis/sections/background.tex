\chapter{Background}
Necessary background for topic, in order to understand the problem, motivation and contribution. Approximately 10 pages.

\section{Javascript Framework}
Eine grundlegende Entscheidungen zu Beginn der Arbeit betraf die Frage, ob ein Javascript-Framework zur Unterstützung bei der Baumvisualisierung eingesetzt werden sollte und, wenn ja, welches. Die Wahl fiel dabei auf D3, weil es neben Unterstützung von Baum- und Graphenstrukturen auch eingebaute Funktionen für Animationen aufweist, welche, wie im Kapitel \todo{Kapitelzahl/name einfügen} beschrieben, eine zentrale Rolle für eine intuitive Bedienung spielen. D3 ist außerdem weit verbreitet, wodurch bei der Entwicklung auf eine reichhaltige Menge von Beispielen und Anregungen zurückgegriffen werden kann.