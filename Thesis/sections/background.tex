\chapter{Background}
Necessary background for topic, in order to understand the problem, motivation and contribution. Approximately 10 pages.
\section{JSON}

\section{D3}\label{sec:framework}
D3 - kurz für Data-Driven Documents - ist eine Javascript Bibliothek zur Visualisierung von Daten auf Internetseiten. Sie ist ein maßgebliches Hilfsmittel im Zuge dieser Arbeit. Das Konzept von D3 ist es, Datensätze mit Elementen eines HTML-Dokuments zu verknüpfen und dadurch großen Einfluss auf die Darstellung dieser Daten zu haben. Dabei werden neben Hilfsfunktionen zum Strukturieren der Datenanzeige auch diverse Möglichkeiten zur Erstellung und Manipulierung von Datenstrukturen angeboten. Da D3 sehr umfangreich ist und die Meisten Anwendungen nicht alle Funktionen benötigen, ist es in Module unterteilt, die einzeln eingebunden werden können. Die für diese Arbeit besonders relevanten Module sind $Selections$, $Hierarchies$, $Shapes$ und $Transitions$.

\subsection{Selections}
$Selections$ sind einer der wichtigsten Bestandteile von D3. Sie dienen dazu HTML-Elemente zu gruppieren, manipulieren und mit Daten zu verknüpfen. Man erstellt sie mit den Befehlen $d3.select(selector)$ oder $d3.selectAll(selector)$, wobei $select$ das erste auf den Übergebenen $selector$ passende und $selectAll$ alle passenden Elemente in eine $Selection$ zusammenführt. Neben den ausgewählten Elementen enthält diese dann auch eine Reihe von Hilfsfunktionen, wie zum Beispiel $selection.data()$.  
\todo{Erklärung von selections und speziell der Funktionen enter(), exit() und data()}
\subsection{Hierarchies}
\todo{Erklärung von heirarchies und wie man damit bäume speichern kann}
\subsection{Shapes}
\subsection{Transitions}
\section{SVG}
\todo{erläuterung von svg elementen, welche elemente im inneren benutzt werden können und ihre bedeutung}